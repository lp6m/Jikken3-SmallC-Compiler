\subsection{プログラム}
プログラム:\textbf{parser.rkt, syntax.rkt}(課題4,6,7で共通)
(parse-file ファイル名)でSmallCのコードを構文解析し、抽象構文木(実際には構造体のリスト)を得る.
\subsection{プログラムの設計方針}
講義ページで配布されていたparser.rktを参考に、アクション文を考えた.構造体はsyntax.rktにある.
すべての木の要素を構造体にするのは効率が悪いため,宣言部分などはリストで持たせるようにした.
\subsection{各部の説明}
以下のようなSmallCコードを用意する.
\lstinputlisting[caption = {sample1.sc},style=smallC]{./code/sample1.sc}
Racketの対話環境でparse-file関数を実行すると以下の通りになる.
\lstinputlisting[caption = {kadai4.scm},style=scheme]{./code/kadai4.scm}
